\documentclass[tikz,border=10pt]{standalone}
%\documentclass[crop, tikz]{standalone}

\usepackage{tikz}

\begin{document}
	
\usetikzlibrary{arrows}
\begin{tikzpicture}[
scale=5,
axis/.style={very thick, ->, >=stealth'},
important line/.style={thick},
dashed line/.style={dashed, thin},
pile/.style={thick, ->, >=stealth', shorten <=2pt, shorten
	>=2pt},
every node/.style={color=black}
]
% axis
\draw[axis] (-0.1,0)  -- (1.1,0) node(xline)[right]
{$G\uparrow/T\downarrow$};
\draw[axis] (0,-0.1) -- (0,1.1) node(yline)[above] {$E$};
% Lines
\draw[important line] (.15,.15) coordinate (A) -- (.85,.85)
coordinate (B) node[right, text width=5em] {$Y^O$};
\draw[important line] (.15,.85) coordinate (C) -- (.85,.15)
coordinate (D) node[right, text width=5em] {$\mathit{NX}=x$};
% Intersection of lines
\fill[red] (intersection cs:
first line={(A) -- (B)},
second line={(C) -- (D)}) coordinate (E) circle (.4pt)
node[above,] {$A$};
% The E point is placed more or less randomly
\fill[red]  (E) +(-.075cm,-.2cm) coordinate (out) circle (.4pt)
node[below left] {$B$};
% Line connecting out and ext balances
\draw [pile] (out) -- (intersection of A--B and out--[shift={(0:1pt)}]out)
coordinate (extbal);
\fill[red] (extbal) circle (.4pt) node[above] {$C$};
% line connecting  out and int balances
\draw [pile] (out) -- (intersection of C--D and out--[shift={(0:1pt)}]out)
coordinate (intbal);
\fill[red] (intbal) circle (.4pt) node[above] {$D$};
% line between out og all balanced out :)
\draw[pile] (out) -- (E);
\end{tikzpicture}
	
\end{document}



