% https://tex.stackexchange.com/a/457351/173708
\documentclass[border=10pt]{standalone}
\usepackage{verbatim}


\begin{comment}

Data file
-------
if you want the data to be saved as you modify this file, add this:

\usepackage{filecontents}

\begin{filecontents*}{events.csv}
X,Y,Event,Date
0,0,Abbotsford,January 1
3,0,Surrey,April 1
5,1,Vancouver,June 1
6,0,New\ Westminster,July 1
10,0,North\ Vancouver,November 1
0.75,4,Ended\ High\ School,Jan 26
3.3,4,Started\ Trade\ School,April 10
5.8,4,Ended\ Trade\ School,June 28
8,4,Started\ Job\ 1,Sept 1
10.8,4,Started\ Job\ 2,Nov 27
\end{filecontents*}
%%%>
\end{comment}

\usepackage{tikz}
\usepackage{pgfplotstable}
\pgfplotstableread[col sep=comma]{./data/events.csv}\data  # in ./data folder
% from https://tex.stackexchange.com/a/445369/121799
\newcommand*{\ReadOutElement}[4]{%
    \pgfplotstablegetelem{#2}{#3}\of{#1}%
    \let#4\pgfplotsretval
}


%only necessary for overset
\usetikzlibrary{positioning}
\usepackage{makecell}%
\usetikzlibrary{patterns}

\begin{document}

    \begin{tikzpicture}[node distance =2mm]

       % draw horizontal line
       \draw (0,0) coordinate (baseLine) -- (11,0);

       % draw vertical lines and label below with months
       \foreach \varXcoord/\varMonth [count=\xx] in {
          0/Jan,
          1/Feb,
          2/Mar,
          3/Apr,
          4/May,
          5/Jun,
          6/Jul,
          7/Aug,
          8/Sep,
          9/Oct,
          10/Nov,
          11/Dec
       }
          \draw (\varXcoord,3pt) -- +(0,-6pt) node (qBaseTick\xx) [below] {\varMonth};

       \fill [pattern=north west lines, pattern color=yellow] (-1,0) rectangle (12.5,4); 

       \node [above left = 2.5 and 1 of baseLine,rotate=90] 
       {
         Location
       };

       \fill [pattern=north west lines, pattern color=orange] (-1,4) rectangle (12.5,8); 

       \node [above left = 7 and 1 of baseLine,rotate=90] 
       {
         Career
       };

      \pgfplotstablegetrowsof{\data}
      \pgfmathtruncatemacro{\rownumber}{\pgfplotsretval-1}
      \foreach \X in {0,...,\rownumber}
      {
          \ReadOutElement{\data}{\X}{X}{\varXcoord}
          \ReadOutElement{\data}{\X}{Y}{\varYcoord}
          \ReadOutElement{\data}{\X}{Event}{\varEvent}
          \ReadOutElement{\data}{\X}{Date}{\varDate}
          \draw[<-] (\varXcoord,0) -- (\varXcoord,\varYcoord+0.5);
          \node [above right = \varYcoord and \varXcoord + 0.5 of baseLine, rotate=45, anchor=south west] {\makecell[l]{\small${\varEvent}$\\\tiny \varDate}};
      }
   \end{tikzpicture}

\end{document}