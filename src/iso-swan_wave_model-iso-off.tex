% layers off

% https://texample.net/tikz/examples/swan-wave-model/
% Author: Marco Miani
% SWAN (developed by SWAN group, TU Delft, The Netherlands) is a wave spectral numerical model.
%For Simlating WAves Nearshore, it is necessary to define spatial grids of
%physical dominant factors (wind friction, dissipation) as well as define a COMPUTATIONAL
%grid on which the model performs its (spectral) calculations: budgeting energy spectra over
%each cell of the (computational) grid. Grids might have different spatial resolution and extension.

\documentclass[12pt]{article}
\usepackage{tikz}
\usetikzlibrary{positioning}


\begin{document}
\pagestyle{empty}

% Define the layers to draw the diagram
\pgfdeclarelayer{background}
\pgfdeclarelayer{foreground}
\pgfsetlayers{background,main,foreground}


\begin{tikzpicture}[scale=.9,every node/.style={minimum size=1cm},on grid]

    \begin{pgfonlayer}{background}
%       \draw [help lines, step=1,color=blue!15, very thin] (-6, 11) grid (10,-7);
    \end{pgfonlayer}          
        
    \begin{pgfonlayer}{foreground}
%        % help guide lines
%        \draw [help lines,dashed] (0,-7) -- (0,11);    
%        \draw [help lines,dashed] (-6,0) -- (10,0);            
%        \node at (9,10) (zero) {(9,10)};        
%        \node at (6,6) (zero) {(6,6)};        
%        \node at (4,4) (zero) {(4,4)};        
%        \node at (-5,4) (zero) {(-5,4)};        
%        \node at (-5,1) (zero) {(-5,1)};        
%        \node at (-5,-2) (zero) {(-5,-2)};        
%        \node at (-5,-5) (zero) {(-5,-5)};             
%        \node at (7,-5) (zero) {(7,-5)};                  
%        \node at (8,-7) (zero) {(8,-7)};                       
%        \node at (0,-7) (zero) {(0,-7)};                                       
    \end{pgfonlayer}          

    % Comp G
    %slanting: production of a set of n 'laminae' to be piled up. N=number of grids.
    \begin{scope}[
        yshift=-83,every node/.append style={
            yslant=0.5,xslant=-1},yslant=0.5,xslant=-1
        ]
        % opacity to prevent graphical interference
        \fill[white,fill opacity=0.9] (0,0) rectangle (5,5);
        \draw[step=5mm, black] (0,0) grid (5,5);        % defining grids
        \draw[step=1mm, red!50,thin] (3,1) grid (4,2);  % Nested Grid
        \draw[black,very thick] (0,0) rectangle (5,5);  % marking borders
        \fill[red] (0.05,0.05) rectangle (0.5,0.5);   % Idem as above, for the n-th grid:
        % add some labels
        \begin{scope}[color=blue,font=\footnotesize]
            \node at (0,0) (a) {(0,0)};
            \node at (5,5) (a) {(5,5)};
            \node at (5,0) (a) {(5,0)};
            \node at (0,5) (a) {(0,5)};   
        \end{scope}     
    \end{scope}
    
    % Bathymetry up
    \begin{scope}[
        yshift=0,every node/.append style={
            yslant=0.5,xslant=-1},yslant=0.5,xslant=-1
        ]
        \fill[white,fill opacity=.9] (0,0) rectangle (5,5);
        \draw[black,very thick] (0,0) rectangle (5,5);
        \draw[step=5mm, black] (0,0) grid (5,5);
    \end{scope}  
    
    % Wind G
    % grid with internal 3x3 of step=10mm
    \begin{scope}[
        yshift=90,every node/.append style={
            yslant=0.5,xslant=-1},yslant=0.5,xslant=-1
        ]
        \fill[white,fill opacity=.9] (0,0) rectangle (5,5);
        \draw[step=10mm, black] (1,1) grid (4,4);
        \draw[black,very thick] (1,1) rectangle (4,4);
        \draw[black,dashed] (0,0) rectangle (5,5);
        
        \node at (1,1) (a) {(1,1)};
    \end{scope}      

    % Friction G
    % grid with green subgrid of 2mm step
    \begin{scope}[
        yshift=170,every node/.append style={
         yslant=0.5,xslant=-1},yslant=0.5,xslant=-1
         ]
        \fill[white,fill opacity=0.6] (0,0) rectangle (5,5);
        \draw[step=10mm, black] (2,2) grid (5,5);
        \draw[step=2mm, green] (2,2) grid (3,3);
        \draw[black,very thick] (2,2) rectangle (5,5);
        \draw[black,dashed] (0,0) rectangle (5,5);
        
        \node at (2,2) (a) {(2,2)};
    \end{scope}    
    
    % bottom grid
    \begin{scope}[
        yshift=-170,every node/.append style={
            yslant=0.5,xslant=-1},yslant=0.5,xslant=-1
        ]
        %marking border
        \draw[black,very thick] (0,0) rectangle (5,5);
        
        %drawing corners (P1,P2, P3): only 3 points needed to define a plane.
        \draw [fill=lime](0,0) circle (.1) ;
        \draw [fill=lime](0,5) circle (.1);
        \draw [fill=lime](5,0) circle (.1);
        \draw [fill=lime](5,5) circle (.1);
        
        %drawing bathymetric hypotetic countours on the bottom grid:    	
        \draw [ultra thick](0,1)   parabola bend (2,2) (5,1)  ;
        \draw [dashed]     (0,1.5) parabola bend (2.5,2.5) (5,1.5) ;
        \draw [dashed]     (0,2)   parabola bend (2.7,2.7) (5,2)  ;
        \draw [dashed]     (0,2.5) parabola bend (3.5,3.5) (5,2.5)  ;
        \draw [dashed]     (0,3.5) parabola bend (2.75,4.5) (5,3.5);
        \draw [dashed]     (0,4)   parabola bend (2.75,4.8) (5,4);
        \draw [dashed]     (0,3)   parabola bend (2.75,3.8) (5,3);
        \draw[-latex,thick](2.8,1) node[right]{$\mathsf{Shoreline}$}
            to[out=180,in=270] (2,1.99);
    \end{scope} %end of drawing grids   
    
    
    % arrows
    %putting arrows and labels:
    \draw[-latex,thick] (6.2,2) node[right]{$\mathsf{Bathymetry (up)}$}
        to[out=180,in=90] (4,2);
    
    \draw[-latex,thick](5.8,-.3)node[right]{$\mathsf{Comp.\ G.}$}
        to[out=180,in=90] (3.9,-1);
    
    \draw[-latex,thick](5.9,5)node[right]{$\mathsf{Wind\ G.}$}
        to[out=180,in=90] (3.6,5);
    
    \draw[-latex,thick](5.9,8.4)node[right]{$\mathsf{Friction\ G.}$}
        to[out=180,in=90] (3.2,8);
    
    \draw[-latex,thick,red](5.3,-4.2)node[right]{$\mathsf{G. Cell}$}
        to[out=180,in=90] (0,-2.5);
    
    \draw[-latex,thick,red](4.3,-1.9)node[right]{$\mathsf{Nested\ G.}$}
        to[out=180,in=90] (2,-.5);
    
    \draw[-latex,thick](4,-6)node[right]{$\mathsf{Batymetry (dn)}$}
        to[out=180,in=90] (2,-5);	    
     

\end{tikzpicture}

\end{document}