% https://tex.stackexchange.com/a/48596/173708
\documentclass{scrartcl}

\usepackage{tikz}
\usetikzlibrary{calc}  

% split figures into pages
\usepackage[active,tightpage]{preview}
\PreviewEnvironment{tikzpicture}
\setlength\PreviewBorder{1pt}%

\begin{document} 

% If their areas of the circle nodes represent some numbers with proportionality 
% then you need to know exactly the radius. The radius depends of minimum width 
% and of \pgflinewidth.
%
% we have : radius = (minimum width + line width) / 2  if inner sep = 0pt
%
% In the next example, I choice first minimum width=2cm then minimum width=2cm,
% line width=5mm and finally line width=5mm,minimum width=2cm-\pgflinewidth 
% with in all cases inner sep= 0 pt.
%

	
\begin{tikzpicture} 
  \draw[help lines,step=0.1,,draw=orange] (0,0) grid (8,1); 
  \draw[help lines] (0,0) grid (8,1);     
  \node[minimum width=2cm,circle,inner sep=0pt,fill=blue!20,fill opacity=.5]{};
  \node[minimum width=2cm,circle,inner sep=0pt,fill=blue!20,fill opacity=.5,
        line width=5mm,draw=gray,opacity=.5] at (3,0){}; 
  \node[circle,inner sep=0pt,fill=blue!20,,fill opacity=.5,
        line width=5mm,draw=gray,opacity=.5,minimum width=2cm-\pgflinewidth]  at (6,0) {}; 
\end{tikzpicture}  

% Now if I want to get three circles with areas equal to pi, 2pi and 3pi 
% I created a macro `def\lw{2mm}` to change quickly the line width in all nodes

\tikzset{myrad/.style 2 args={circle,inner sep=0pt,minimum width=(2*(sqrt(#1)*1 cm ) - \pgflinewidth,fill=#2,draw=#2,fill opacity=.5,opacity=.8}}    

\begin{tikzpicture} 
	\def\lw{2mm}
	\draw[help lines,step=0.1,,draw=orange] (0,0) grid (8,1); 
	\draw[help lines] (0,0) grid (8,1);     
	\node[line width=\lw,myrad={1}{blue!20}]  at (0,0) {1}; 
	\node[line width=\lw,myrad={2}{red!20}]  at (3,0) {2};
	\node[line width=\lw, myrad={3}{green!20}]  at (7,0) {3};   
\end{tikzpicture}  

% Finally If you want nodes with areas equal to 1 cm^2, 2 cm^2 and 3 cm^2 : 
% I change the line width for the second group of nodes

\begin{tikzpicture} 
	\def\lw{2mm}
	\draw[help lines,step=0.1,,draw=orange] (0,0) grid (8,1); 
	\draw[help lines] (0,0) grid (8,1);     
	\node[line width=\lw,myrad={1}{blue!20}]  at (0,0) {1}; 
	\node[line width=\lw,myrad={2}{red!20}]  at (3,0) {2};
	\node[line width=\lw, myrad={3}{green!20}]  at (7,0) {3};   
\end{tikzpicture}    

\begin{tikzpicture} 
	\def\lw{5mm}
	\draw[help lines,step=0.1,,draw=orange] (0,0) grid (8,1); 
	\draw[help lines] (0,0) grid (8,1);     
	\node[line width=\lw,myrad={1}{blue!20}]  at (0,0) {1}; 
	\node[line width=\lw,myrad={2}{red!20}]  at (3,0) {2};
	\node[line width=\lw, myrad={3}{green!20}]  at (7,0) {3};   
\end{tikzpicture}

%To avoid this kind of problem, we can use circles instead of circle nodes. But we need to adjust the radius wit the pgflinewidth. In the next example,I want a radius = 2cm so I need to use : radius=2cm-0.5\pgflinewidth. Then I need to create a node with the same dimensions.
%
%Like the question about node and rectangle here, we can associate a node to the shape The main problem : we can't use scale but it's more easy to place a label.

\tikzset{set node/.style={insert path={% 
	\pgfextra{% 
		\node[inner sep=0pt,outer sep = 0pt,draw=black, % draw= none only to show what I do
		circle,
		minimum width=2*\pgfkeysvalueof{/tikz/x radius}+0.5\pgflinewidth](#1) {};
}}}}

\begin{tikzpicture}
	\draw[help lines] (-3,-3) grid (3,3);
	\draw[blue,line width=5mm,opacity=.2] (0,0) circle [radius=2cm-0.5\pgflinewidth,set node=C1]  ; 
	\draw[thick,->] (3,-3) -- (C1.east);
\end{tikzpicture}

\end{document} 