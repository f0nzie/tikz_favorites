% https://gist.github.com/MatteoRagni/678b1ac217c59db8ced872871da5e323
\documentclass[tikz]{standalone}

\usepackage{pgfplots}
\usepackage{amsmath}
\usepackage{tikz-3dplot}

\pgfplotsset{compat=1.14}
\tikzset{>=latex}

\begin{document}

% !TEX root = ../main.tex
\tdplotsetmaincoords{60}{100}
\begin{tikzpicture}[scale=1.5,tdplot_main_coords]
  % DEFINITIONS
  % figOmegaRotation : Value for the Omega angle
    \pgfmathsetmacro{\figOmega}{33}
  % figIRotation : value for inclination of the orbital plane
    \pgfmathsetmacro{\figIRotation}{30}
  % \figAEll : minimum value for ellipse
    \pgfmathsetmacro{\figAEll}{2.5}
  % \figBEll : minimum value for ellipse
    \pgfmathsetmacro{\figBEll}{3}
  % figYOrigin : focus of the ellipse with 2.5 and 3
    \pgfmathsetmacro{\figYOrigin}{sqrt(\figBEll^2 - \figAEll^2)}
    % sqrt(b^2 - a^2)

  % COMMANDS
  % \positionOnEl{\storing}{\quadrant}{\x}
  % Saves into \storing the position \y on the ellipse
  % specified by \figAEll and \figBEll. \quadrant should help
  % in defining the quadrant (must be +1 or -1)
  \newcommand{\positionXOnEl}[3]{%
    \pgfmathsetmacro{#1}%
      {\figYOrigin+#2*%
      \figBEll*((1-#3^2/\figAEll^2)^2)^(0.25)}}
  \newcommand{\positionYOnEl}[3]{%
    \pgfmathsetmacro{#1}%
      {#2*((\figBEll^2-\figYOrigin^2+2*\figYOrigin*#3-#3^2)^2)^(0.25)*(\figAEll/\figBEll)}}


  \positionYOnEl{\ellPositY}{-1}{\figBEll}
  \positionYOnEl{\ellPositYY}{1}{0}

  \pgfmathsetmacro{\xstartV}{cos(atan(\figBEll/\ellPositY))*0.5}
  \pgfmathsetmacro{\ystartV}{sin(atan(\figBEll/\ellPositY))*0.5}

  % Origin Position for the drawing
  \coordinate (O) at (0, 0, 0);

  % REFERENCE FRAME
  % x axis
  \draw[tdplot_main_coords, very thin]
    (-4.5, 0, 0) -- (-0.5, 0, 0);
  % y axis
  \draw[tdplot_main_coords, very thin]
    (0, -2.1, 0) -- (0, -0.5, 0);
  \draw[tdplot_main_coords, ->, very thin]
    (0, 0.5, 0) -- (0, 5, 0)
    node [anchor=west] {$y$};


  % EQUATORIAL PLANE
  \draw[tdplot_main_coords, very thin,gray]
    (4, 4.7, 0) -- (4, -2, 0) --
    node[anchor=south, rotate=54] {\footnotesize equatorial plane}
    (-4,-2, 0) -- (-4, 4.7, 0) -- (4, 4.7, 0);
  % Equatorial plane Ellipses
  \draw[tdplot_main_coords, very thin, dashed, gray!50]
    (0, \figYOrigin, 0) ellipse ({\figAEll} and {\figBEll});



  % Rotazione sulla linea dei nodi
  \tdplotsetrotatedcoords{0}{0}{\figOmega}

  % Ellipses in equatorial plane
  \draw[tdplot_rotated_coords, fill, color=gray!40, opacity=0.1]
    (0, \figYOrigin, 0) ellipse ({\figAEll} and {\figBEll});
  \draw[tdplot_rotated_coords, very thin, dashed]
    (0, \figYOrigin, 0) ellipse ({\figAEll} and {\figBEll});
  %  reference on rotated ellipse
  \draw[tdplot_rotated_coords, very thin]
    (0, -2.1, 0) -- (0, -0.5, 0);
  \draw[tdplot_rotated_coords, ->, very thin]
    (0, 0.5, 0) -- (0, 5, 0)
    node [anchor=west] {$y'$};

  \shade[tdplot_rotated_coords, ball color = green!40, opacity = 0.8]
    (0, 0, 0) circle (0.5cm);

  % Rotation on ascension
  \tdplotsetrotatedcoords{90+\figOmega}{-\figIRotation}{-90}
  % Orbit ellipse
  \draw[tdplot_rotated_coords, fill, color=gray!40, opacity=0.3]
    (0, \figYOrigin, 0) ellipse ({\figAEll} and {\figBEll});
  \draw[tdplot_rotated_coords, very thin]
    (0, \figYOrigin, 0) ellipse ({\figAEll} and {\figBEll});



  % Part of trajectory
  \draw[tdplot_rotated_coords, very thick]
    (\ellPositYY, 0, 0)
    arc (-33.5:154:{\figAEll} and {\figBEll});

  % Line of Nodes
  \draw[tdplot_rotated_coords]
    (0.5, 0, 0) -- (4, 0, 0)
    node [anchor=north west,fill=white] {\footnotesize line of nodes};
  % Line of apsis
  \draw[tdplot_rotated_coords, ->, very thin]
    (0, 0.5, 0) -- (0, 5, 0)
    node[anchor=south west] {\footnotesize line of apsis};

  % V vector (solved through ellipse equation)
  \draw[tdplot_rotated_coords,->,thick]
    (-\xstartV, -\ystartV, 0) -- (\ellPositY, \figBEll, 0);
  \shade[tdplot_rotated_coords,ball color = green]
    (\ellPositY, \figBEll, 0) circle (0.05cm)
    node[anchor=south east, color=green!40!black!50] {\footnotesize debris};

  % Planet
  % Equatorial circle in
  \draw[tdplot_rotated_coords, very thin]
    (0, 0, 0) circle (0.5);
  \draw[tdplot_main_coords, very thin, gray]
    (0, 0, 0) circle (0.5);
  % x axis in original reference frame
  % redesegn for z-buffer
  \draw[tdplot_main_coords, ->, very thin]
    (0.75, 0, 0) -- (4.5, 0, 0)
    node [anchor=north] {$x$};
    % z axis
  \draw[tdplot_main_coords, ->, very thin]
    (0, 0, 0.5) -- (0, 0, 2)
    node[anchor=south] {$z$};

  % OMEGA
  \tdplotdrawarc[tdplot_main_coords, color=red, ->]
    {(0,0,0)}{2.5}{0}{\figOmega}{anchor=north}{$\Omega$}
  % omega
  \tdplotdrawarc[tdplot_rotated_coords, color=red, ->]
    {(0,0,0)}{1.8}{0}{90}{anchor=west}{$\omega$}
  \tdplotdrawarc[tdplot_rotated_coords, color=red, ->]
    {(0,0,0)}{1.8}{90}{126.8}{anchor=south}{$v$}
  % n vector
  \draw[tdplot_rotated_coords, ->, color=red, pos=0.8]
    (0.5, 0, 0) --  node [anchor=north east, pos=0.6] {$n$} (\ellPositYY, 0, 0);
  \draw[tdplot_rotated_coords, ->, color=red]
    (0, 0.5, 0) --  node [anchor=south east, pos=0.85] {$e$}
    (0, 3 + \figYOrigin, 0);

  % piano per i
  \shade[tdplot_rotated_coords,ball color = blue]
    (0, \figYOrigin + \figBEll, 0) circle (0.05cm)
    node[anchor=west, color=blue] {\footnotesize apoapsis};
  \shade[tdplot_rotated_coords,ball color = red] (\ellPositYY, 0, 0) circle
    (0.05cm) node[anchor=west, color=red] {\footnotesize ascending node};

  \draw[tdplot_rotated_coords] (0,0,0) circle (0.5);

  \tdplotsetrotatedcoords{\figOmega}{90}{0}
  % i angle
  \tdplotdrawarc[tdplot_rotated_coords, color=red, ->]
    {(0, 0, 0)}{\figYOrigin+\figBEll}{90}{118}{anchor=west}{$i$}
  % Orbital Plane Label
  \node[tdplot_main_coords,rotate=33.5,anchor=east,color=gray]
    at (-4.8,-0.3,0) {\footnotesize orbital plane};

  % Planet

\end{tikzpicture}





\end{document}