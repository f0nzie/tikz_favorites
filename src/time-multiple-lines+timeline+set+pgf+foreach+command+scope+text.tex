% https://tex.stackexchange.com/a/448018/173708
\documentclass[12pt]{article}
\usepackage[a4paper, margin=1cm]{geometry}
\usepackage{tikz}
\pagestyle{empty}

\newcommand{\anno}{1} % starting year
\newcommand{\target}{31} % ending year
\newcommand{\alto}{3} % height

\tikzset{
    start date/.code args = {#1/#2}{
        \def\dstart{#1}
        \def\mstart{#2}
    },
    end date/.code args = {#1/#2}{
        \def\dend{#1}
        \def\mend{#2}
    },
    event color/.style = {
        fill=#1!50, draw=#1,
    },
}

\pgfmathsetmacro{\myend}{\target+1-\anno}
\pgfmathsetmacro{\myspacing}{16/(\target-1-\anno)}


\newcommand{\eventpoint}[2][]{
    \begin{scope}[#1]
    \pgfmathsetmacro{\mmstart}{((13-\mstart)*4)}
    \pgfmathsetmacro{\mmend}{((13-\mend)*4)}
    \ifnum\mstart=\mend
       \filldraw (\dstart, \mmstart ) rectangle (\dend, \mmend+1) node [font=\scriptsize, text centered, midway, inner sep=0pt] {#2};
    \else
       \filldraw (\dstart, \mmstart ) rectangle (31, \mmstart+1) node [font=\scriptsize, text centered, midway, inner sep=0pt] {#2};
      \filldraw (0, \mmend ) rectangle (\dend, \mmend+1) node [font=\scriptsize, text centered, midway, inner sep=0pt] {#2};
    \fi
    \end{scope}
}

\begin{document}
\centering
\begin{tikzpicture}[x=\myspacing cm,y=5mm]

	%creating the sublines needed and formating them
	\foreach \y in {0,4,8,12,16,20,24,28,32,36,40,44,48}{
	\draw[|->, -latex] (-.5,\y) -- (\myend+.5,\y);
	\path (0,0) -- (0,\alto);
	\foreach \x [evaluate=\x as \day using int(\x)] in {0,10,20,30}{
	    \draw (\x,\y) node[below=7pt,font=\footnotesize] {$\day$};
	    \draw (\x,\y -.2) -- (\x,\y +.2);
	    \draw[loosely dotted] (\x,\y +.2) -- (\x,\y+ \alto-0.5);
	}
	\foreach \tick in {0,...,\myend}{
	    \draw (\tick,\y +.1) -- (\tick,\y -.1);
	}
	}
	
	%trying to add the events
	\eventpoint[start date=15/7,end date=25/7,event color=green]{test}
	\eventpoint[start date=17/8,end date=19/9,event color=red,yshift=5pt]{test 2} % shift just to show you can hook into standard tikz keys to change the event's features ad-hoc

\end{tikzpicture}
\end{document}