% Highlighting elements in matrices
% Author: Stefan Kottwitz
% \documentclass{article}
\documentclass[preview]{standalone}
\usepackage{tikz}
\usetikzlibrary{fit}
\tikzset{%
  highlight/.style={rectangle,rounded corners,fill=red!15,draw,fill opacity=0.5,thick,inner sep=0pt}
}
\newcommand{\tikzmark}[2]{\tikz[overlay,remember picture,baseline=(#1.base)] \node (#1) {#2};}
%
\newcommand{\Highlight}[1][submatrix]{%
    \tikz[overlay,remember picture]{
    \node[highlight,fit=(left.north west) (right.south east)] (#1) {};}
}

\begin{document}

\[
  M = \left(\begin{array}{*5{c}}
    \tikzmark{left}{1} & 2 & 3 & 4 & 5\\
    6 & 7 & 8 & 9 & 10 \\
    11 & 12 & \tikzmark{right}{13} & 14 & 15 \\
    16 & 17 & 18 & 19 & 20
  \end{array}\right)
  \Highlight[first]
  \qquad
  M^T = \left(\begin{array}{*5{c}}
    \tikzmark{left}{1} & 6 & 11 & 16 \\
    2 & 7 & 12 & 17 \\
    3 & 8 & \tikzmark{right}{13} & 18 \\
    4 & 9 & 14 & 19 \\
    5 & 10 & 15 & 20
  \end{array}\right)
\]
\Highlight[second]

\tikz[overlay,remember picture] {
  \draw[->,thick,red,dashed] (first) -- (second) node [pos=0.66,above] {Transpose};
  \node[above of=first] {$N$};
  \node[above of=second] {$N^T$};
}
\end{document}
