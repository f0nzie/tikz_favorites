\documentclass{standalone}

\usepackage{tikz}

\usetikzlibrary{arrows}
\usepackage{verbatim}

\begin{comment}
:Title: Radix-2 FFT signal flow
:Slug: radix2fft
:Tags: Foreach

Radix-2 signal flow graph for a 16 point fast Fourier transform (FFT).
This diagram is quite complex. However, the most difficult part is keeping
track of all the indexes. The ``foreach`` command is used extensively to
get compact code.

| Source: The diagram was inspired by content on `this web page`__.

.. __: http://www.ece.uvic.ca/499/2004a/group05/html/background.html

\end{comment}

\begin{document}
\pagestyle{empty}

\tikzstyle{n}= [circle, fill, minimum size=4pt,inner sep=0pt, outer sep=0pt]
\tikzstyle{mul} = [circle,draw,inner sep=-1pt]

% Define two helper counters
\newcounter{x}\newcounter{y}
\begin{tikzpicture}[yscale=0.5, xscale=1.2, node distance=0.3cm, auto]
    % The strategy is to create nodes with names: N-column-row
    % Input nodes are named N-0-0 ... N-0-15
    % Output nodes are named N-10-0 ... N-10-15

    % Draw inputs
    \foreach \y in {0,...,15}
        \node[n, pin={[pin edge={latex'-,black}]left:$x(\y)$}]
              (N-0-\y) at (0,-\y) {};
    % Draw outputs
    \foreach \y / \idx in {0/0,1/8,2/4,3/12,4/2,5/10,6,7/14,
                           8/1,9,10/5,11/13,12/3,13/11,14/7,15}
        \node[n, pin={[pin edge={-latex',black}]right:$X(\idx)$}]
              (N-10-\y) at (7,-\y) {};
   % draw connector nodes
    \foreach \y in {0,...,15}
        \foreach \x / \c in {1/1,2/3,3/4,4/6,5/7,6/9}
            \node[n, name=N-\c-\y] at (\x,-\y) {};
    % draw x nodes
    \foreach \y in {0,...,15}
        \foreach \x / \c  in {1/2,4/5,7/8}
            \node[mul, right of=N-\x-\y] (N-\c-\y) {${\times}$};

    % horizontal connections
    % Note the use of simple counter arithmetics to get correct
    % indexes.
    \foreach \y in {0,...,15}
        \foreach \x in {0,1,3,4,6,7,9}
        {
            \setcounter{x}{\x}\stepcounter{x}
            \path (N-\x-\y) edge[-] (N-\arabic{x}-\y);
       }
    % Draw the W_16 coefficients
    \setcounter{y}{0}
    \foreach \i / \j in {0/0,1/0,2/0,3/0,4/0,5/0,6/0,7/0,
                            0/1,1/1,2/1,3/1,4/1,5/1,6/1,7/1}
    {
        \path (N-2-\arabic{y}) edge[-] node {\tiny $W^{\i\cdot\j}_{16}$}
                (N-3-\arabic{y});
        \stepcounter{y}
    }
    % Draw the W_8 coefficients
    \setcounter{y}{0}
    \foreach \i / \j in {0/0,1/0,2/0,3/0,0/1,1/1,2/1,3/1,
                         0/0,1/0,2/0,3/0,0/1,1/1,2/1,3/1}
    {
        \path (N-5-\arabic{y}) edge[-] node {\tiny $W^{\i\cdot\j}_{8}$}
              (N-6-\arabic{y});
        \addtocounter{y}{1}
    }

    % Draw the W_4 coefficients
    \setcounter{y}{0}
    \foreach \i / \j in {0/0,1/0,0/1,1/1,0/0,1/0,0/1,1/1,
                            0/0,1/0,0/1,1/1,0/0,1/0,0/1,1/1}
    {
        \path (N-8-\arabic{y}) edge[-] node {\tiny $W^{\i\cdot\j}_{4}$}
              (N-9-\arabic{y});
        \stepcounter{y}
    }
    % Connect nodes
    \foreach \sourcey / \desty in {0/8,1/9,2/10,3/11,
                                   4/12,5/13,6/14,7/15,
                                   8/0,9/1,10/2,11/3,
                                   12/4,13/5,14/6,15/7}
       \path (N-0-\sourcey.east) edge[-] (N-1-\desty.west);
    \foreach \sourcey / \desty in {0/4,1/5,2/6,3/7,
                                   4/0,5/1,6/2,7/3,
                                   8/12,9/13,10/14,11/15,
                                   12/8,13/9,14/10,15/11}
        \path (N-3-\sourcey.east) edge[-] (N-4-\desty.west);
    \foreach \sourcey / \desty in {0/2,1/3,2/0,3/1,
                                   4/6,5/7,6/4,7/5,
                                   8/10,9/11,10/8,11/9,
                                   12/14,13/15,14/12,15/13}
        \path (N-6-\sourcey.east) edge[-] (N-7-\desty.west);
    \foreach \sourcey / \desty in {0/1,1/0,2/3,3/2,
                                   4/5,5/4,6/7,7/6,
                                   8/9,9/8,10/11,11/10,
                                   12/13,13/12,14/15,15/14}
        \path (N-9-\sourcey.east) edge[-] (N-10-\desty.west);

\end{tikzpicture}

\end{document}